\usepackage{bm}
% 画像
\usepackage[dvipdfmx]{graphicx}
% 修飾
\usepackage{ascmac}
\usepackage{mathrsfs}
% 図式
\usepackage{tikz}
\usetikzlibrary{arrows.meta}

\theoremstyle{definition}
\newtheorem{theorem}{定理}
\newtheorem{proposition}[theorem]{命題}
\newtheorem{lemma}[theorem]{補題}
\newtheorem{corollary}[theorem]{系}
\newtheorem{example}[theorem]{例}
\newtheorem{definition}[theorem]{定義}
\newtheorem{remark}[theorem]{注意}
\newtheorem{guide}[theorem]{参考}
\newtheorem{structure}[theorem]{構成}
\newtheorem{axiom}[theorem]{公理}
\newtheorem{exercise}{演習}
\numberwithin{theorem}{subsection}  % 定理番号を「定理2.3.1」のように印刷
\numberwithin{exercise}{subsection}
\numberwithin{equation}{section} % 式番号を「(3.5.1)」のように印刷

\DeclareMathOperator{\Ima}{Im}
\DeclareMathOperator{\Ker}{Ker}
\DeclareMathOperator{\Cok}{Cok}
\DeclareMathOperator{\Tr}{Tr}
\DeclareMathOperator{\Hom}{Hom}
\newcommand{\spanning}[2]{\text{span}\{#1\:;\:#2\}}

\everymath{\displaystyle} %数式をすべてDisplaystyleで表示

\begin{document}

\title{Basic Calculas}
\author{hayami-m}
\date{4/14}

\maketitle

\begin{itembox}[l]{4/13}
  \begin{align*}
    \lim_{n\to\infty}\frac{1}{n^2}\int_{1}^{2^n}t^{-1+\frac{1}{n^2}}\cos{t}dt=0
  \end{align*}
  であることを示せ.
\end{itembox}

\begin{itembox}[l]{4/15}
  calculate \(\sum_{n=0}^{\infty}\frac{(-1)^n}{3n+1}\).  
\end{itembox}

\begin{itembox}[l]{4/21}
  正方行列を一つ作り、以下の問いに答えよ。
  \begin{enumerate}
    \item 固有方程式を求め, 固有値とその重複度を求めよ. 固有ベクトルと固有空間の字数も答えよ.
    \item 最小多項式を求めよ.
    \item DeterminantとTraceを求めよ.
    \item 対角化可能性を判定せよ.
    \item ジョルダン標準形を求めよ.
    \item もう一つ行列を作り, 相似かどうか判定せよ.
  \end{enumerate}
\end{itembox}

\begin{itembox}[l]{Stirling's approximation}
  次の式を示せ.
  \begin{enumerate}
    \item (Wallis' product) \(\prod_{n=0}^{\infty}\frac{(2n)^2}{(2n-1)(2n+1)}=\frac{\pi}{2}\)
    \item (Stirling's approximation) \(\lim_{n\to\infty}\frac{n!e^n}{n^n\sqrt{2\pi{n}}}=1\)
  \end{enumerate}
\end{itembox}

\begin{itembox}[l]{計算練習(III-4.8.2)}
  次の積分\(I_n\)に対する漸化式を作れ.
  \begin{enumerate}
    \item \(I_n=\int\tan^n{x}dx\)
    \item \(I_n=\int\sin^n{x}\cos^m{x}dx\)
    \item \(I_n=\int(\log{x})^{n}dx\)
    \item \(I_n=\int{x}^{m}(\log{x})^{n}dx\)
    \item \(I_n=\int(\arcsin{x})^{n}dx\)
  \end{enumerate}
\end{itembox}

\end{document}